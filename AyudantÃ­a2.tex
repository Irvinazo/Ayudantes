\documentclass{article}
\usepackage[utf8]{inputenc}

\usepackage[utf8]{inputenc}
\usepackage[spanish,es-tabla,es-nodecimaldot]{babel}
\usepackage{amsmath,amsthm,amsfonts,amssymb,mathtools,dsfont,mathrsfs}
\usepackage{enumerate,graphicx,xcolor}
\usepackage{lmodern}
\usepackage[T1]{fontenc}
\usepackage[left=2cm,top=2.5cm,right=2cm,bottom=2.5cm]{geometry}
\usepackage[activate={true,nocompatibility},final,tracking=true,kerning=true,spacing=true,factor=1100,stretch=10,shrink=10]{microtype}
\usepackage{hyperref}


%\DeclarePairedDelimiter{\norm}{\lVert}{\rVert}




\newcommand{\N}{\mathbb{N}}
\newcommand{\R}{\mathbb R}
\newcommand{\Z}{\mathbb Z}
\newcommand{\Rbar}{\overline{\mathbb R}}
\newcommand{\F}{\mathscr F}
\newcommand{\A}{\mathscr A}
\newcommand{\To}{\Rightarrow}
\newcommand{\C}{\mathscr C}
\newcommand{\La}{\mathscr L_A}
\newcommand{\B}{\mathcal B}
\newcommand{\Q}{\mathbb Q}
\renewcommand{\epsilon}{\varepsilon}
\renewcommand{\L}{\mathcal L}
\renewcommand{\d}{\mathrm d}
\newcommand{\abs}[1]{\left| #1 \right|}
\newcommand{\pts}[1]{\left( #1 \right)}
\newcommand{\norm}[1]{\left\lVert#1\right\rVert}
\renewcommand{\P}[1]{\mathbb P\left( #1 \right)}
\newcommand{\E}[1]{\mathbb E \left( #1 \right)}


\newcommand{\ols}[1]{\mskip.5\thinmuskip\overline{\mskip-.5\thinmuskip {#1} \mskip-.5\thinmuskip}\mskip.5\thinmuskip} % overline short
\newcommand{\olsi}[1]{\,\overline{\!{#1}}} % overline short italic
\makeatletter
\newcommand\closure[1]{
  \tctestifnum{\count@stringtoks{#1}>1} %checks if number of chars in arg > 1 (including '\')
  {\ols{#1}} %if arg is longer than just one char, e.g. \mathbb{Q}, \mathbb{F},...
  {\olsi{#1}} %if arg is just one char, e.g. K, L,...
}
% FROM TOKCYCLE:
\long\def\count@stringtoks#1{\tc@earg\count@toks{\string#1}}
\long\def\count@toks#1{\the\numexpr-1\count@@toks#1.\tc@endcnt}
\long\def\count@@toks#1#2\tc@endcnt{+1\tc@ifempty{#2}{\relax}{\count@@toks#2\tc@endcnt}}
\def\tc@ifempty#1{\tc@testxifx{\expandafter\relax\detokenize{#1}\relax}}
\long\def\tc@earg#1#2{\expandafter#1\expandafter{#2}}
\long\def\tctestifnum#1{\tctestifcon{\ifnum#1\relax}}
\long\def\tctestifcon#1{#1\expandafter\tc@exfirst\else\expandafter\tc@exsecond\fi}
\long\def\tc@testxifx{\tc@earg\tctestifx}
\long\def\tctestifx#1{\tctestifcon{\ifx#1}}
\long\def\tc@exfirst#1#2{#1}
\long\def\tc@exsecond#1#2{#2}
\makeatother

\newtheorem{lemma}{Lema}
\newtheorem{theorem}{Teorema}

\setlength\parindent{0pt}
\setlength\parskip{4pt}

% \geometry{
%  a4paper,
%  total={170mm,257mm},
%  left=20mm,
%  top=20mm,
%  }
 \geometry{
 a4paper,
 total={100mm,157mm},
 left=10mm,
 top=10mm,
 }

 \theoremstyle{definition}
\newtheorem{definition}{Definición}



\title{Elementos de probabilidad y estadística. Ayudantía 2.}
\date{9 de febrero de 2024}



\begin{document}

\maketitle


\begin{enumerate}
    \item (El problema del cumpleaños) Una clase tiene $n$ estudiantes. ¿Cuál es la
    probabilidad de que al menos dos tengan el mismo cumpleaños? (ignora los gemelos
    y los años bisiestos). Calcula la probabilidad para $n=10,15,22,30$ y 40.

    \item Un 14 de febrero en un lugar de Guanajuato, hay 5 parejas celebrando.
    Un operativo policiaco llega al lugar e inspecciona a cuatro personas al azar,
    ¿cuál es la probabilidad de que entre los inspeccionados haya al menos una pareja?

    \item P. R. de Montmort, escribió en 1708 un problema sobre el juego francés
    Jeu de Boules.  El objetivo del juego es  tirar pelotas de metal hacia una pelota 
    objetivo.  La pelota más cercana gana.  Supón que dos jugadores Irving y Samuel son igual
    de buenos en este juego. Irving lanza una pelota, y Samuel lanza dos. ¿Cuál es la
    probabilidad de que Irving gane?

    \item La NASA está desarrollando dos transbordadores espaciales ultra secretos,
    uno tiene dos motores y el otro tiene cuatro.  Todos los motores son idénticos 
    y tienen la misma probabilidad de fallar.  Cada uno está diseñado para volar si 
    al menos 
    \item $N$ personas, incluyendo Barbie y Ken, están sentadas en una mesa redonda con $N$ asientos. 
    ¿Cuál es la probabilidad de que Barbie y Ken no se sienten juntos? ¿Cuál es la probabilidad de que los dos 
    se sienten forzosamente juntos?
    \item \textbf{(La biblioteca de Babel):} Jorge Luis Borges a través de este libro nos cuenta una historia en donde 
    el lector se da cuenta eventualmente que la biblioteca en cuestión contiene todo posible libro. Literalmente, todo posible libro. 
    Así, por ejemplo, la librería debería tener un libro que consistiera en solamente la letra `A'. Otro podría 
    tener solamente `abcabcabc...' en todas sus páginas, y así sucesivamente.\\

    Cada uno de los libros tiene 410 páginas, cada página tiene 40 líneas y cada línea tiene 80 carateres. 
    Hay exactamente 25 símbolos, incluyendo el espacio y los símbolos de puntuación. \\

    Suponiendo que todo posible libro está incluído en la biblioteca y que no hay duplicados, ¿cuántos libros 
    distintos tiene la librería?\\

    Algunos astrónomos estiman que hay alrededor de $10^{87}$ partículas sub-atómicas en el universo.
    ¿Hay más o menos libros en la biblioteca de Babel que partículas en el universo? ¿Cómo se compara el número 
    de libros con un googol $\left(10^{100}\right)$? ¿Y con un googolplex $\left(10^{10^{100}}\right)$?

    \item \textbf{(El Problema de las cajas de cerillos de Banach\footnote[1]{Stefan Banch fue un matemático polaco. En los años previos a la Segunda 
    Guerra Mundial, él y sus colegas solían reunirse en el Scottish Coffee Shop en Lvov, Polonia (hoy Leópolis, Ucrania), a platicar sobre matemáticas.
    Tenían entre todos un libro al que solían llamar el Scottish Book, en el cual escribían ejercicios retadores. Este fue el último 
    problema del libro.})}: Un cierto profesor fumador y de mirada perdida 
    carga dos cajitas de cerillos, una en el lado izquierdo de su abrigo y la otra en el derecho. Cada vez 
    que él enciende su pipa elige una de las dos cajitas al azar y toma un cerillo de ella. Eventualmente, se 
    da cuenta que la cajita de cerillos que ha elegido está vacía. Si ambas cajitas tienen inicialmente el mismo número 
    de cerillos, digamos $n$, ¿cuál es la probabilidad de que queden exactamente $k$ cerillos en la otra caja 
    cuando la primera se ha vaciado?

\end{enumerate}




\end{document}
